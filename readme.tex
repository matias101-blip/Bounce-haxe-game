% Created 2024-09-11 mié 19:50
% Intended LaTeX compiler: pdflatex
\documentclass[11pt]{article}
\usepackage[utf8]{inputenc}
\usepackage[T1]{fontenc}
\usepackage{graphicx}
\usepackage{longtable}
\usepackage{wrapfig}
\usepackage{rotating}
\usepackage[normalem]{ulem}
\usepackage{amsmath}
\usepackage{amssymb}
\usepackage{capt-of}
\usepackage{hyperref}
\usepackage[spanish]{babel}
\author{Murcia Matias, Chacha Hernan, Gomez Emiro}
\date{\today}
\title{Documentacion del proyecto Bounce haxe}
\hypersetup{
 pdfauthor={Murcia Matias, Chacha Hernan, Gomez Emiro},
 pdftitle={Documentacion del proyecto Bounce haxe},
 pdfkeywords={},
 pdfsubject={},
 pdfcreator={Emacs 29.4 (Org mode 9.6.24)}, 
 pdflang={Spanish}}
\begin{document}

\maketitle
\tableofcontents

\section{Bounce juego - Haxe con Flixel}
\label{sec:orgcfd5092}

\begin{figure}[htbp]
\centering
\includegraphics[width=4cm]{./img/haxeflixel.png}
\caption{\label{fig:HaxeFlixel}HaxeFlixel Icon}
\end{figure}

Para este proyecto se usará como lenguaje de programación
el lenguaje de alto nivel \guillemotleft{}Haxe\guillemotright{} con la librería \guillemotleft{}Flixel\guillemotright{}

El juego es una recreación o inspiración del juego \guillemotleft{}Bounce Tales\guillemotright{}

\subsection{Creación del proyecto}
\label{sec:org4b04bba}
El proyecto está desarrollado en \guillemotleft{}Flixel\guillemotright{} la cual es una librería para
el lenguaje \guillemotleft{}Haxe\guillemotright{}, de esta forma se puede compilar para html5 (Web), Neko (para escritorio).

Primero se instala el lenguaje \guillemotleft{}Haxe\guillemotright{} en el siguiente enlace llevará a la página oficial
de descarga \href{https://haxe.org/download/}{Haxe Download}.

\subsubsection{Flixel, OpenFL, Lime}
\label{sec:org1c868bc}
Las librerías que se usarán son Flixel, OpenFL, Lime
Las cuales se instalan de la siguiente forma:
\begin{verbatim}
#Instala la librería de Flixel
haxelib install fhttps://github.com/matias101-blip/Bounce-haxe-gamelixel

#Instala la librería de Lime
haxelib install lime

#Instala la librería de OpenFL
haxelib install openfl
\end{verbatim}
Con las librerías instaladas podemos construir el proyecto.

\subsection{Testeo y construcción}
\label{sec:org53c0995}
Para testar el juego es necesario clonar el repositorio de GitHub o bajar el ZIP,
en una consola o terminal ejecutar:
\begin{verbatim}
#Para HTML5, este se ejecuta en el navegador.
haxelib run lime test html5

#Para el escritorio, se ejecuta de forma nativa.
haxelib run lime test neko
\end{verbatim}
\subsection{Desarrollo del player}
\label{sec:org4d51fda}
A continuación, en este apartado se explicará el desarrollo del proyecto.
\subsubsection{Objetivos}
\label{sec:orga025d57}
\subsubsection{Player}
\label{sec:org325fd3b}
\subsection{Desarrollo de niveles}
\label{sec:org9d97279}
\end{document}
